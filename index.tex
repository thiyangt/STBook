% Options for packages loaded elsewhere
\PassOptionsToPackage{unicode}{hyperref}
\PassOptionsToPackage{hyphens}{url}
\PassOptionsToPackage{dvipsnames,svgnames,x11names}{xcolor}
%
\documentclass[
  11pt,
  a4paper,
]{report}

\usepackage{amsmath,amssymb}
\usepackage{setspace}
\usepackage{iftex}
\ifPDFTeX
  \usepackage[T1]{fontenc}
  \usepackage[utf8]{inputenc}
  \usepackage{textcomp} % provide euro and other symbols
\else % if luatex or xetex
  \usepackage{unicode-math}
  \defaultfontfeatures{Scale=MatchLowercase}
  \defaultfontfeatures[\rmfamily]{Ligatures=TeX,Scale=1}
\fi
\usepackage{lmodern}
\ifPDFTeX\else  
    % xetex/luatex font selection
\fi
% Use upquote if available, for straight quotes in verbatim environments
\IfFileExists{upquote.sty}{\usepackage{upquote}}{}
\IfFileExists{microtype.sty}{% use microtype if available
  \usepackage[]{microtype}
  \UseMicrotypeSet[protrusion]{basicmath} % disable protrusion for tt fonts
}{}
\makeatletter
\@ifundefined{KOMAClassName}{% if non-KOMA class
  \IfFileExists{parskip.sty}{%
    \usepackage{parskip}
  }{% else
    \setlength{\parindent}{0pt}
    \setlength{\parskip}{6pt plus 2pt minus 1pt}}
}{% if KOMA class
  \KOMAoptions{parskip=half}}
\makeatother
\usepackage{xcolor}
\usepackage[top=2.5cm,bottom=2.5cm,left=2.5cm,right=2.5cm]{geometry}
\setlength{\emergencystretch}{3em} % prevent overfull lines
\setcounter{secnumdepth}{2}

\usepackage{color}
\usepackage{fancyvrb}
\newcommand{\VerbBar}{|}
\newcommand{\VERB}{\Verb[commandchars=\\\{\}]}
\DefineVerbatimEnvironment{Highlighting}{Verbatim}{commandchars=\\\{\}}
% Add ',fontsize=\small' for more characters per line
\usepackage{framed}
\definecolor{shadecolor}{RGB}{241,243,245}
\newenvironment{Shaded}{\begin{snugshade}}{\end{snugshade}}
\newcommand{\AlertTok}[1]{\textcolor[rgb]{0.68,0.00,0.00}{#1}}
\newcommand{\AnnotationTok}[1]{\textcolor[rgb]{0.37,0.37,0.37}{#1}}
\newcommand{\AttributeTok}[1]{\textcolor[rgb]{0.40,0.45,0.13}{#1}}
\newcommand{\BaseNTok}[1]{\textcolor[rgb]{0.68,0.00,0.00}{#1}}
\newcommand{\BuiltInTok}[1]{\textcolor[rgb]{0.00,0.23,0.31}{#1}}
\newcommand{\CharTok}[1]{\textcolor[rgb]{0.13,0.47,0.30}{#1}}
\newcommand{\CommentTok}[1]{\textcolor[rgb]{0.37,0.37,0.37}{#1}}
\newcommand{\CommentVarTok}[1]{\textcolor[rgb]{0.37,0.37,0.37}{\textit{#1}}}
\newcommand{\ConstantTok}[1]{\textcolor[rgb]{0.56,0.35,0.01}{#1}}
\newcommand{\ControlFlowTok}[1]{\textcolor[rgb]{0.00,0.23,0.31}{\textbf{#1}}}
\newcommand{\DataTypeTok}[1]{\textcolor[rgb]{0.68,0.00,0.00}{#1}}
\newcommand{\DecValTok}[1]{\textcolor[rgb]{0.68,0.00,0.00}{#1}}
\newcommand{\DocumentationTok}[1]{\textcolor[rgb]{0.37,0.37,0.37}{\textit{#1}}}
\newcommand{\ErrorTok}[1]{\textcolor[rgb]{0.68,0.00,0.00}{#1}}
\newcommand{\ExtensionTok}[1]{\textcolor[rgb]{0.00,0.23,0.31}{#1}}
\newcommand{\FloatTok}[1]{\textcolor[rgb]{0.68,0.00,0.00}{#1}}
\newcommand{\FunctionTok}[1]{\textcolor[rgb]{0.28,0.35,0.67}{#1}}
\newcommand{\ImportTok}[1]{\textcolor[rgb]{0.00,0.46,0.62}{#1}}
\newcommand{\InformationTok}[1]{\textcolor[rgb]{0.37,0.37,0.37}{#1}}
\newcommand{\KeywordTok}[1]{\textcolor[rgb]{0.00,0.23,0.31}{\textbf{#1}}}
\newcommand{\NormalTok}[1]{\textcolor[rgb]{0.00,0.23,0.31}{#1}}
\newcommand{\OperatorTok}[1]{\textcolor[rgb]{0.37,0.37,0.37}{#1}}
\newcommand{\OtherTok}[1]{\textcolor[rgb]{0.00,0.23,0.31}{#1}}
\newcommand{\PreprocessorTok}[1]{\textcolor[rgb]{0.68,0.00,0.00}{#1}}
\newcommand{\RegionMarkerTok}[1]{\textcolor[rgb]{0.00,0.23,0.31}{#1}}
\newcommand{\SpecialCharTok}[1]{\textcolor[rgb]{0.37,0.37,0.37}{#1}}
\newcommand{\SpecialStringTok}[1]{\textcolor[rgb]{0.13,0.47,0.30}{#1}}
\newcommand{\StringTok}[1]{\textcolor[rgb]{0.13,0.47,0.30}{#1}}
\newcommand{\VariableTok}[1]{\textcolor[rgb]{0.07,0.07,0.07}{#1}}
\newcommand{\VerbatimStringTok}[1]{\textcolor[rgb]{0.13,0.47,0.30}{#1}}
\newcommand{\WarningTok}[1]{\textcolor[rgb]{0.37,0.37,0.37}{\textit{#1}}}

\providecommand{\tightlist}{%
  \setlength{\itemsep}{0pt}\setlength{\parskip}{0pt}}\usepackage{longtable,booktabs,array}
\usepackage{calc} % for calculating minipage widths
% Correct order of tables after \paragraph or \subparagraph
\usepackage{etoolbox}
\makeatletter
\patchcmd\longtable{\par}{\if@noskipsec\mbox{}\fi\par}{}{}
\makeatother
% Allow footnotes in longtable head/foot
\IfFileExists{footnotehyper.sty}{\usepackage{footnotehyper}}{\usepackage{footnote}}
\makesavenoteenv{longtable}
\usepackage{graphicx}
\makeatletter
\newsavebox\pandoc@box
\newcommand*\pandocbounded[1]{% scales image to fit in text height/width
  \sbox\pandoc@box{#1}%
  \Gscale@div\@tempa{\textheight}{\dimexpr\ht\pandoc@box+\dp\pandoc@box\relax}%
  \Gscale@div\@tempb{\linewidth}{\wd\pandoc@box}%
  \ifdim\@tempb\p@<\@tempa\p@\let\@tempa\@tempb\fi% select the smaller of both
  \ifdim\@tempa\p@<\p@\scalebox{\@tempa}{\usebox\pandoc@box}%
  \else\usebox{\pandoc@box}%
  \fi%
}
% Set default figure placement to htbp
\def\fps@figure{htbp}
\makeatother

\usepackage{booktabs}
\usepackage{longtable}
\usepackage{array}
\usepackage{multirow}
\usepackage{wrapfig}
\usepackage{float}
\usepackage{colortbl}
\usepackage{pdflscape}
\usepackage{tabu}
\usepackage{threeparttable}
\usepackage{threeparttablex}
\usepackage[normalem]{ulem}
\usepackage{makecell}
\usepackage{xcolor}
\makeatletter
\@ifpackageloaded{bookmark}{}{\usepackage{bookmark}}
\makeatother
\makeatletter
\@ifpackageloaded{caption}{}{\usepackage{caption}}
\AtBeginDocument{%
\ifdefined\contentsname
  \renewcommand*\contentsname{Table of contents}
\else
  \newcommand\contentsname{Table of contents}
\fi
\ifdefined\listfigurename
  \renewcommand*\listfigurename{List of Figures}
\else
  \newcommand\listfigurename{List of Figures}
\fi
\ifdefined\listtablename
  \renewcommand*\listtablename{List of Tables}
\else
  \newcommand\listtablename{List of Tables}
\fi
\ifdefined\figurename
  \renewcommand*\figurename{Figure}
\else
  \newcommand\figurename{Figure}
\fi
\ifdefined\tablename
  \renewcommand*\tablename{Table}
\else
  \newcommand\tablename{Table}
\fi
}
\@ifpackageloaded{float}{}{\usepackage{float}}
\floatstyle{ruled}
\@ifundefined{c@chapter}{\newfloat{codelisting}{h}{lop}}{\newfloat{codelisting}{h}{lop}[chapter]}
\floatname{codelisting}{Listing}
\newcommand*\listoflistings{\listof{codelisting}{List of Listings}}
\makeatother
\makeatletter
\makeatother
\makeatletter
\@ifpackageloaded{caption}{}{\usepackage{caption}}
\@ifpackageloaded{subcaption}{}{\usepackage{subcaption}}
\makeatother

\usepackage[style=authoryear-comp,]{biblatex}
\addbibresource{thesisrefs.bib}
\usepackage{bookmark}

\IfFileExists{xurl.sty}{\usepackage{xurl}}{} % add URL line breaks if available
\urlstyle{same} % disable monospaced font for URLs
\hypersetup{
  pdftitle={Spatio-Temporal Data Analysis},
  pdfauthor={Thiyanga S. Talagala},
  colorlinks=true,
  linkcolor={blue},
  filecolor={Maroon},
  citecolor={Blue},
  urlcolor={Blue},
  pdfcreator={LaTeX via pandoc}}

%% CAPTIONS
\usepackage{caption}
\DeclareCaptionStyle{italic}[justification=centering]
 {labelfont={bf},textfont={it},labelsep=colon}
\captionsetup[figure]{style=italic,format=hang,singlelinecheck=true}
\captionsetup[table]{style=italic,format=hang,singlelinecheck=true}

%% FONT
\usepackage{bera}
\usepackage[charter]{mathdesign}
%\usepackage[scale=0.9]{sourcecodepro}
\usepackage[lf,t]{FiraSans}
\usepackage{fontawesome}

%% HEADERS AND FOOTERS
\usepackage{fancyhdr}
\pagestyle{fancy}
\rfoot{\Large\sffamily\raisebox{-0.1cm}{\textbf{\thepage}}}
\makeatletter
\lhead{\textsf{\expandafter{\@title}}}
\makeatother
\rhead{}
\cfoot{}
\setlength{\headheight}{15pt}
\renewcommand{\headrulewidth}{0.4pt}
\renewcommand{\footrulewidth}{0.4pt}
\fancypagestyle{plain}{%
\fancyhf{} % clear all header and footer fields
\fancyfoot[C]{\sffamily\thepage} % except the center
\renewcommand{\headrulewidth}{0pt}
\renewcommand{\footrulewidth}{0pt}}

%% MATHS
\usepackage{bm,amsmath}
\allowdisplaybreaks

%% GRAPHICS
\makeatletter
\def\fps@figure{htbp}
\makeatother
\setcounter{topnumber}{2}
\setcounter{bottomnumber}{2}
\setcounter{totalnumber}{4}
\renewcommand{\topfraction}{0.85}
\renewcommand{\bottomfraction}{0.85}
\renewcommand{\textfraction}{0.15}
\renewcommand{\floatpagefraction}{0.8}
\graphicspath{{figures/}}

%% SECTION TITLES
\usepackage[compact,sf,bf]{titlesec}
\titleformat*{\section}{\Large\sf\bfseries}
\titleformat*{\subsection}{\large\sf\bfseries}
\titleformat*{\subsubsection}{\sf\bfseries}
\titlespacing{\section}{0pt}{*5}{*1}
\titlespacing{\subsection}{0pt}{*2}{*0.2}
\titlespacing{\subsubsection}{0pt}{*1}{*0.1}

%% TABLES
\usepackage{booktabs,tabu}

%% BIBLIOGRAPHY.

\makeatletter
\@ifpackageloaded{biblatex}{
\ExecuteBibliographyOptions{bibencoding=utf8,minnames=1,maxnames=3, maxbibnames=99,dashed=false,terseinits=true,giveninits=true,uniquename=false,uniquelist=false,doi=false, isbn=false,url=true,sortcites=false}
\DeclareFieldFormat{url}{\texttt{\url{#1}}}
\DeclareFieldFormat[article]{pages}{#1}
\DeclareFieldFormat[inproceedings]{pages}{\lowercase{pp.}#1}
\DeclareFieldFormat[incollection]{pages}{\lowercase{pp.}#1}
\DeclareFieldFormat[article]{volume}{\mkbibbold{#1}}
\DeclareFieldFormat[article]{number}{\mkbibparens{#1}}
\DeclareFieldFormat[article]{title}{\MakeCapital{#1}}
\DeclareFieldFormat[article]{url}{}
\DeclareFieldFormat[inproceedings]{title}{#1}
\DeclareFieldFormat{shorthandwidth}{#1}
\usepackage{xpatch}
\xpatchbibmacro{volume+number+eid}{\setunit*{\adddot}}{}{}{}
% Remove In: for an article.
\renewbibmacro{in:}{%
  \ifentrytype{article}{}{%
  \printtext{\bibstring{in}\intitlepunct}}}
\AtEveryBibitem{\clearfield{month}}
\AtEveryCitekey{\clearfield{month}}
\DeclareDelimFormat[cbx@textcite]{nameyeardelim}{\addspace}
\renewcommand*{\finalnamedelim}{\addspace\&\space}
}{}
\makeatother


\hypersetup{
     pdfcreator={Quarto -> pandoc -> LaTeX -> pdf}
}


%% PAGE BREAKING to avoid widows and orphans
\clubpenalty = 2000
\widowpenalty = 2000
\usepackage{microtype}
\def\maketitle{
\pagenumbering{roman}
{\sf\thispagestyle{empty}%
  \null\vskip-.4cm%
  \centerline{\includegraphics[width=12cm]{monash-logo}}
  \vspace*{4cm}
  \begin{center}\fontsize{24}{28}\sf
     \textbf{Spatio-Temporal Data Analysis}\\[2cm]
     \fontsize{18}{20}\sf Thiyanga S. Talagala\\[0.2cm]
     \fontsize{13}{15}\sf B.Sc. (Hons), University of Tangambalanga
     \vfill
     \fontsize{13}{15}\sf A thesis submitted for the degree of\\ Doctor
of Philosophy\\ at Monash University in \number\the\year\\
     Department of Econometrics \& Business Statistics
  \end{center}
  \newpage\mbox{}\thispagestyle{empty}\newpage
}
}

% Title and date

\title{Spatio-Temporal Data Analysis}
\date{}
\begin{document}
\maketitle

\renewcommand*\contentsname{Table of contents}
{
\hypersetup{linkcolor=}
\setcounter{tocdepth}{1}
\tableofcontents
}

\setstretch{1.5}
\bookmarksetup{startatroot}

\chapter*{Copyright notice}\label{copyright-notice}
\addcontentsline{toc}{chapter}{Copyright notice}

\markboth{Copyright notice}{Copyright notice}

Produced on 21 September 2025.

© Thiyanga S. Talagala (2025).

\bookmarksetup{startatroot}

\chapter*{Abstract}\label{abstract}
\addcontentsline{toc}{chapter}{Abstract}

\markboth{Abstract}{Abstract}

The abstract should outline the main approach and findings of the thesis
and must not be more than 500 words.

\bookmarksetup{startatroot}

\chapter*{Declaration}\label{declaration}
\addcontentsline{toc}{chapter}{Declaration}

\markboth{Declaration}{Declaration}

\begin{quote}
Use only one of the following declarations (Standard thesis or Thesis
including published works declaration) and remove the other.
\end{quote}

\subsection*{Standard thesis}\label{standard-thesis}
\addcontentsline{toc}{subsection}{Standard thesis}

This thesis is an original work of my research and contains no material
which has been accepted for the award of any other degree or diploma at
any university or equivalent institution and that, to the best of my
knowledge and belief, this thesis contains no material previously
published or written by another person, except where due reference is
made in the text of the thesis.

Student name:

Student signature:

Date:

\subsubsection*{Publications during
enrolment}\label{publications-during-enrolment}
\addcontentsline{toc}{subsubsection}{Publications during enrolment}

\begin{quote}
Remove this section if you do not have publications.
\end{quote}

The material in Chapter~\ref{sec-intro} has been submitted to the
journal \emph{Journal of Impossible Results} for possible publication.

The contribution in \textbf{?@sec-litreview} of this thesis was
presented in the International Symposium on Nonsense held in Dublin,
Ireland, in July 2022.

\subsubsection*{Reproducibility
statement}\label{reproducibility-statement}
\addcontentsline{toc}{subsubsection}{Reproducibility statement}

This thesis is written using Quarto with renv \autocite{renv} to create
a reproducible environment. All materials (including the data sets and
source files) required to reproduce this document can be found at the
Github repository
\href{https://github.com/SusanSu/thesis}{\texttt{github.com/SusanSu/thesis}}.

This work is licensed under a
\href{http://creativecommons.org/licenses/by-nc-sa/4.0/}{Creative
Commons Attribution-NonCommercial-ShareAlike 4.0 International License}.

\subsection*{Thesis including published works
declaration}\label{thesis-including-published-works-declaration}
\addcontentsline{toc}{subsection}{Thesis including published works
declaration}

I hereby declare that this thesis contains no material which has been
accepted for the award of any other degree or diploma at any university
or equivalent institution and that, to the best of my knowledge and
belief, this thesis contains no material previously published or written
by another person, except where due reference is made in the text of the
thesis.

This thesis includes ?? original papers published in peer reviewed
journals and ?? submitted publications. The core theme of the thesis is
??. The ideas, development and writing up of all the papers in the
thesis were the principal responsibility of myself, the student, working
within the Department of Econometrics \& Business Statistics under the
supervision of ??

(The inclusion of co-authors reflects the fact that the work came from
active collaboration between researchers and acknowledges input into
team-based research.)

In the case of (??insert chapter numbers) my contribution to the work
involved the following:

\begingroup\fontsize{10}{12}\selectfont

\resizebox{\ifdim\width>\linewidth\linewidth\else\width\fi}{!}{
\begin{tabu} to \linewidth {>{\raggedleft\arraybackslash}p{1.2cm}>{\raggedright\arraybackslash}p{2.6cm}>{\raggedright}X>{\raggedright\arraybackslash}p{2.6cm}>{\raggedright\arraybackslash}p{2.6cm}>{\raggedright\arraybackslash}p{2.6cm}}
\toprule
\multicolumn{1}{>{\raggedright\arraybackslash}p{1.2cm}}{\textbf{Thesis chapter}} & \multicolumn{1}{>{\raggedright\arraybackslash}p{2.6cm}}{\textbf{Publication title}} & \multicolumn{1}{l}{\textbf{Status}} & \multicolumn{1}{>{\raggedright\arraybackslash}p{2.6cm}}{\textbf{Nature and \% of student contribution}} & \multicolumn{1}{>{\raggedright\arraybackslash}p{2.6cm}}{\textbf{Nature and \% of coauthors' contribution}} & \multicolumn{1}{>{\raggedright\arraybackslash}p{2.6cm}}{\textbf{Coauthors are Monash students}}\\
\midrule
2 & The life cycle of Mongolian crickets & Submitted & Concept and data analysis, writing first draft: 60\% & Shu Xu, input into manuscript: 25\%; Eddie Betts, input into manuscript: 15\% & Shu Xu: No; Eddie Betts: Yes\\
\bottomrule
\end{tabu}}
\endgroup{}

I have / have not renumbered sections of submitted or published papers
in order to generate a consistent presentation within the thesis.

Student name:

Student signature:

Date:

I hereby certify that the above declaration correctly reflects the
nature and extent of the student's and co-authors' contributions to this
work. In instances where I am not the responsible author I have
consulted with the responsible author to agree on the respective
contributions of the authors.

Main Supervisor name:

Main Supervisor signature:

Date:

\bookmarksetup{startatroot}

\chapter*{Acknowledgements}\label{acknowledgements}
\addcontentsline{toc}{chapter}{Acknowledgements}

\markboth{Acknowledgements}{Acknowledgements}

I would like to thank my pet goldfish for \ldots{}

\begin{quote}
In accordance with Chapter 7.1.4 of the research degrees handbook, if
you have engaged the services of a~professional~editor, you
must~provide~their name~and a brief description of the service rendered.
If the professional editor's current or former area of academic
specialisation is similar your own, this too should be stated as it may
suggest to examiners that the editor's advice to the student has
extended beyond guidance on English expression to affect the substance
and structure of the thesis.
\end{quote}

\begin{quote}
If you have used generative artificial intelligence (AI) technologies,
you must include a written acknowledgment of the use and its extent.
Your acknowledgement should at a minimum specify which technology was
used, include explicit description on how the information was generated,
and explain how the output was used in your work. Below is a suggested
format:
\end{quote}

\begin{quote}
``I acknowledge the use of {[}insert AI system(s) and link{]} to
{[}specific use of generative artificial intelligence{]}. The output
from these was used to {[}explain use{]}.''
\end{quote}

\begin{quote}
Free text section for you to record your acknowledgment and gratitude
for the more general academic input and support such as financial
support from grants and scholarships and the non-academic support you
have received during the course of your enrolment. If you are a
recipient of the ``Australian Government Research Training Program
Scholarship'', you are required to include the following statement:
\end{quote}

\begin{quote}
\begin{quote}
``This research was supported by an Australian Government Research
Training Program (RTP) Scholarship.''
\end{quote}
\end{quote}

\begin{quote}
You may also wish to acknowledge significant and substantial
contribution made by others to the research, work and writing
represented and/or reported in the thesis. These could include
significant contributions to: the conception and design of the project;
non-routine technical work; analysis and interpretation of research
data; drafting significant parts of the work or critically revising it
to contribute to the interpretation.
\end{quote}

\clearpage\pagenumbering{arabic}\setcounter{page}{1}

\bookmarksetup{startatroot}

\chapter{Introduction}\label{sec-intro}

In statistics and data science, datasets can take different forms
depending on how they are collected and organized. Understanding the
type of data is crucial because it guides the choice of appropriate
analytical methods.

\section{Cross-sectional data}\label{cross-sectional-data}

Data collected at a single point in time across multiple units (e.g.,
households, firms, individuals).

\textbf{Example:} household income survey conducted in 2025.

\textbf{Assumption:} Each observation (e.g., each household, individual,
firm) is assumed to be unrelated to the others.

In practice, this assumption can be violated if:

\begin{itemize}
\item
  There's clustering (e.g., individuals from the same village may be
  correlated).
\item
  There's spatial correlation (e.g., nearby locations may be similar).
\item
  There's hidden time effects (if data were not truly collected at the
  same time).
\end{itemize}

\section{Time series data}\label{time-series-data}

A time series is a sequence of observations taken sequentially in time.
The data may consist of one variable (univariate time series) or
multiple variables (multivariate time series) observed over regular or
irregular time intervals.

\textbf{Examples:}

Univariate: Monthly rainfall in Colombo from 2000--2025.

Multivariate: Monthly rainfall, temperature, and humidity in Colombo
from 2000--2025.

\section{Spatial data}\label{spatial-data}

Data linked to a geographical location or space.

Example: soil pH levels measured across different districts in Sri
Lanka.

\section{Spatio-temporal data}\label{spatio-temporal-data}

Data that varies across both space and time.

Example: daily dengue cases recorded across different districts over
several years.

\section{Longitudinal data (Repeated
cross-sections)}\label{longitudinal-data-repeated-cross-sections}

Longitudinal data refer to data collected through repeated measurements
over time. The measurements may be taken on the same units (e.g.,
following the same households each year) or on different units at
different time points (e.g., different random samples of households each
year).

\textbf{Example (different random samples of households each year)}

Suppose a national health survey is conducted every 5 years (2000, 2005,
2010, 2015, 2020). Each time, a new random sample of 5,000 households is
selected.

In 2000 → Households A, B, C, \ldots{}

In 2005 → Households X, Y, Z, \ldots{}

In 2010 → Households P, Q, R, \ldots{}

Here, the same households are not followed across time, but the survey
is still longitudinal, since measurements are taken repeatedly over time
to study population-level changes (e.g., trends in obesity, smoking
rates, or income inequality).

\section{Panel data}\label{panel-data}

Panel data are a special case of longitudinal data, where the same units
are observed consistently across multiple time periods. This allows
analysts to study both within-unit dynamics (how a given unit changes
over time) and between-unit differences.

In finance and econometric modelling, panel data is widely used because
it captures both the cross-sectional dimension (different firms,
individuals, or markets) and the time dimension (repeated observations).

In Panel data and Longitudinal data, which combines cross-sectional and
time-series data, allows for the examination of both ``within-behavior''
and ``between-behavior'' effects.

\textbf{Example (Country-level Panel Data)}

Suppose you collect data on GDP growth rates for 50 countries from
2000--2020.

\textbf{1. Country-specific behavior (within a country over time)}

You can see how Sri Lanka's GDP growth changed year by year.

Example:

\begin{itemize}
\tightlist
\item
  was there a slowdown after the 2008 global crisis, followed by
  recovery?
\end{itemize}

\textbf{2. cross-country and temporal effects (Between countries over
time)}

You can compare trends across countries.

Example:

\begin{itemize}
\item
  Did most countries experience a dip in 2008--2009 due to the financial
  crisis?
\item
  Do developing countries generally grow faster than developed countries
  over these 20 years?
\end{itemize}

\bookmarksetup{startatroot}

\chapter{Visualising Time Series
Data}\label{visualising-time-series-data}

\section{Time series}\label{time-series}

A time series is a sequence of observations recorded in time order. The
time intervals between observations can be regular (e.g., daily,
monthly, yearly) or irregular (e.g., magnitude of a earthquake at a
particular location).

\section{Frequency of a time series (Seasonal
periods)}\label{frequency-of-a-time-series-seasonal-periods}

Number of observations per natural time interval (Usually year, but
sometimes a week, a day, an hour)

\subsection{Single Seasonality}\label{single-seasonality}

The time series exhibits one repeating pattern at a fixed frequency.

Example:

Monthly sales that peak every December (annual seasonality).

\begin{longtable}[]{@{}lr@{}}
\toprule\noalign{}
Data & Frequency \\
\midrule\noalign{}
\endhead
\bottomrule\noalign{}
\endlastfoot
Annual & 1 \\
Quarterly & 4 \\
Monthly & 12 \\
Weekly & 52 \\
\end{longtable}

\subsection{Multiple Seasonality}\label{multiple-seasonality}

The time series exhibits more than one repeating pattern at different
frequencies simultaneously.

Example:

Hourly electricity demand with a daily pattern (peaks every day at
certain hours), a weekly pattern (weekdays vs weekends).

Website traffic with hourly variation and seasonal holiday peaks.

\begin{longtable}[]{@{}cccccc@{}}
\caption{Time Unit Frequencies}\tabularnewline
\toprule\noalign{}
Data & Minute & Hour & Day & Week & Year \\
\midrule\noalign{}
\endfirsthead
\toprule\noalign{}
Data & Minute & Hour & Day & Week & Year \\
\midrule\noalign{}
\endhead
\bottomrule\noalign{}
\endlastfoot
Daily & NA & NA & NA & 7 & 365.25 \\
Hourly & NA & NA & 24 & 168 & 8766.00 \\
Half-hourly & NA & NA & 48 & 336 & 17532.00 \\
Minutes & 60 & 1440 & 1440 & 10080 & 525960.00 \\
Seconds & 60 & 3600 & 86400 & 604800 & 31557600.00 \\
\end{longtable}

\section{\texorpdfstring{\texttt{DataFrame} for time series
data}{DataFrame for time series data}}\label{dataframe-for-time-series-data}

When your DataFrame represents a time series, the index is usually the
date or time, allowing pandas to:

\begin{itemize}
\item
  Plot time series easily
\item
  Resample or aggregate data by time
\item
  Compute rolling statistics
\end{itemize}

\begin{Shaded}
\begin{Highlighting}[]
\CommentTok{\# Import pandas}
\CommentTok{\#py {-}m pip install pandas}
\ImportTok{import}\NormalTok{ pandas }\ImportTok{as}\NormalTok{ pd}

\CommentTok{\# Define data}
\NormalTok{value }\OperatorTok{=}\NormalTok{ [}\DecValTok{100}\NormalTok{, }\DecValTok{250}\NormalTok{, }\DecValTok{78}\NormalTok{, }\DecValTok{300}\NormalTok{, }\DecValTok{500}\NormalTok{]}
\NormalTok{time }\OperatorTok{=} \BuiltInTok{list}\NormalTok{(}\BuiltInTok{range}\NormalTok{(}\DecValTok{2015}\NormalTok{, }\DecValTok{2020}\NormalTok{))}

\CommentTok{\# Create DataFrame}
\NormalTok{df }\OperatorTok{=}\NormalTok{ pd.DataFrame(\{}\StringTok{"Year"}\NormalTok{: time, }\StringTok{"Observation"}\NormalTok{: value\})}

\CommentTok{\# Set \textquotesingle{}Year\textquotesingle{} as index}
\NormalTok{df.set\_index(}\StringTok{"Year"}\NormalTok{, inplace}\OperatorTok{=}\VariableTok{True}\NormalTok{)}

\CommentTok{\# Display the DataFrame}
\BuiltInTok{print}\NormalTok{(df)}
\end{Highlighting}
\end{Shaded}

\begin{verbatim}
      Observation
Year             
2015          100
2016          250
2017           78
2018          300
2019          500
\end{verbatim}

For data collected more often than once a year (e.g., monthly, weekly,
or daily), it's important to tell the computer that the index represents
time. We do this by converting the index to a time or date type using a
time-class function. This helps us sort, select, and analyze the data
correctly over time.

\begin{Shaded}
\begin{Highlighting}[]
\CommentTok{\# Sample monthly data}
\NormalTok{data }\OperatorTok{=}\NormalTok{ \{}
    \StringTok{"Month"}\NormalTok{: pd.date\_range(start}\OperatorTok{=}\StringTok{"2025{-}01{-}01"}\NormalTok{, periods}\OperatorTok{=}\DecValTok{6}\NormalTok{, freq}\OperatorTok{=}\StringTok{"M"}\NormalTok{),  }\CommentTok{\# 6 months}
    \StringTok{"Sales"}\NormalTok{: [}\DecValTok{120}\NormalTok{, }\DecValTok{150}\NormalTok{, }\DecValTok{170}\NormalTok{, }\DecValTok{130}\NormalTok{, }\DecValTok{180}\NormalTok{, }\DecValTok{200}\NormalTok{]}
\NormalTok{\}}

\CommentTok{\# Create DataFrame}
\NormalTok{z }\OperatorTok{=}\NormalTok{ pd.DataFrame(data)}

\CommentTok{\# Format Month as "Year Month" (e.g., "2025 Jan")}
\NormalTok{z[}\StringTok{"Month"}\NormalTok{] }\OperatorTok{=}\NormalTok{ z[}\StringTok{"Month"}\NormalTok{].dt.strftime(}\StringTok{"\%Y \%b"}\NormalTok{)}

\CommentTok{\# Set Month as index}
\NormalTok{z.set\_index(}\StringTok{"Month"}\NormalTok{, inplace}\OperatorTok{=}\VariableTok{True}\NormalTok{)}

\CommentTok{\# Display the DataFrame}
\BuiltInTok{print}\NormalTok{(z)}
\end{Highlighting}
\end{Shaded}

\begin{verbatim}
          Sales
Month          
2025 Jan    120
2025 Feb    150
2025 Mar    170
2025 Apr    130
2025 May    180
2025 Jun    200
\end{verbatim}

\section{Time series visualisation using grammar of
graphics}\label{time-series-visualisation-using-grammar-of-graphics}

\subsection{Introduction to ``grammar of graphics'' with
plotnine}\label{introduction-to-grammar-of-graphics-with-plotnine}

\bookmarksetup{startatroot}

\chapter*{Bibliography}\label{bibliography}
\addcontentsline{toc}{chapter}{Bibliography}

\markboth{Bibliography}{Bibliography}

\printbibliography[heading=none]





\end{document}
